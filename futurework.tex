%!TEX root = main.tex

\chapter{Future Work}

\textbf{Extending the functionality of Jantu:} \\
Because of the way LIDA is structured it is by design easy to extend the behaviour of the system. By adding new feature detectors one can extend the sensing capabilities of the internal LIDA system to more and more complex thoughts and concepts. This also enables you to select more context appropriate actions to perform because you have more information available to you. More actions can also be implemented, and different ways to perform the actions. 

\textbf{Integrating more complex build order systems:} \\
If the program is going to consistently perform well versus different opponents, then a more complex build order prediction and selection system has to be integrated into the bot. Build orders are a very important part of a starcraft game because it goes hand in hand with the players overall strategies and tactics. Build orders are the backbone in any strategy since it unlocks different unit compositions that can be used. 

\textbf{Opponent modeling:} \\
Predicting what you opponent is going to do is a very important part of playing any multi-player game that isn't fully observable. Being prepared for an attack, or even countering it with an earlier attack or better unit composition is a good way to win a match, and starcraft is no exception to that. LIDA has features for long term memory, and opponent modeling could possibly be implemented using that and learning.

\textbf{Learning:} \\
In the current state of the LIDA framework, at the time of writing this report, the LIDA framework does not support learning as it has not been fully implemented yet. But in a future version of the framework when this has been implemented there are many ways to improve upon the system using learning. Learning can be used to discover connections and opponent specific situations. 
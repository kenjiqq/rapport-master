%!TEX root = main.tex

\chapter{Introduction}
In this chapter we introduce our project, as well as the background and motivation for doing this. Section \ref{sec:background} presents the background and our motivation for the project, a short introduction to Starcraft and the API we will use and describes what we will do in this project and our main goals. Section \ref{sec:contributions} contains the main contributors to the project as well as our supervisors. Section \ref{sec:goals} explains the goals we hope to accomplish during the project. Section \ref{sec:structure} introduces the structure of this report.

\section{Background and Motivation}
\label{sec:background}
It is our opinion that the study of cognitive architectures are woefully underrepresented in the field of artificial intelligence, and also in the domain of computer games. We believe that the use of cognitive architectures should be further explored to attain a better understanding of the gains from using a cognitive architecture.

\subsection{The Problem}
Up to 2003, there wasn't much academic research done in the domain of real-time strategy games, until Michael Burro published an influental paper in which he outlined various reasons for why this should change.\cite{buro2003real} Before his paper, most research on strategic reasoning was done in the domain of turn-based board games like chess and checkers, which in itself was impressive as it progressed to the point where computers are now able to beat top level human players in real-time matches.\cite{campbell2002deep} The most obvious difference between turn-based games like that, and real-time strategy games like the ones Burro was calling attention to, is that most real-time strategy games are only partially observable and partially deterministic. This makes for much more interesting problems when it comes to reasoning, in our opinion.

After Burro's paper several research platforms emerged, including OpenRTS\cite{buro2003orts} and Wargus\cite{wargus}. Wargus is an open source WarCraft 2 clone, which uses the Lua scripting language for its agent creation, while OpenRTS is a platform originally created for AI research. These platforms have some limitations, however, because they are mostly developed by hobbyists and not professional game development companies, and therefore lack both polish and wide-spread usage outside of academic environments. They also lack a lot in the way of individual unit management, path finding, etc.

\subsubsection{BWAPI}
Therefore a project called BWAPI, which lets people easily create agents that play a commercial real-time strategy game called StarCraft: Brood War, has emerged and gotten quite popular.\cite{bwapi} There are several competitions held each year, where programmers pit their BWAPI-based agents against eachother, often for money.\cite{sscait} These artificial agents still need a lot of work before they can match human players, however.\cite{eisbotvsfong}

And while just winning the game is certainly an interesting and worthy goal, it is our opinion that more closely matching human-level playing is much more interesting. According to Arrabales et al, it is still much more engaging for human players to play with other humans than with artificial agents.\cite{arrabales2009gamechars} We therefore think that developing human-like intelligence for playing computer games is something that should be researched.

\subsubsection{Cognitive architectures}
Utilizing cognitive architectures is one possible way of solving this. Cognitive architectures are architectures that base themselves on an understanding of how human cognition work, and the logical conclusion would therefore be that using this for implementing an agent would lead to human-like behaviour. There are several cognitive models, and architectures that try to implement these computationally, a select few who are presented in this thesis.

\section{Name}
\label{sec:name}
The name {\em Jantu} is hindi, and is translated as ``Pertaining to the merely sentient part of a creature, as distinguished from the intellectual, rational, or spiritual part; as, the animal passions or appetites.''\cite{hindijantu}, which reflects what we are trying to achieve with our work in this project.

\section{Contributions}
\label{sec:contributions}
Firstly we would also like to thank our supervisors; Helge Langseth and Anders Kofod-Petersen. We would also like to thank the Cognitive Computing Research Group at the University of Memphis for their LIDA framework which is the basis of our project. Finally we would like to thank the developers of the BWAPI project, and in particular Adam Heinermann, as well as the rest of the members in the \#BWAPI IRC channel on Freenode.

\section{Project goals}
\label{sec:goals}
First and foremost we set out to create a proof of concept agent that could play the game StarCraft: Brood War, built using a cognitive architecture.

Secondly we wanted to see what kind of challenges present themselves when one implements an agent like this, and where there is room for improvement, and what the future might look like.

\section{Report Structure}
\label{sec:structure}
This report is structured into four chapters:
\begin{itemize}
\item Chapter 1: \textbf{Introduction} \\
This chapter describes the motivation and goal of the project as well as who
contributed and the general structure of the report.
\item Chapter 2: \textbf{Theory} \\
Here we present the domain we're working in; the real-time computer strategy game StarCraft: Brood War. We also present the cognitive architecture we're using.
\item Chapter 3: \textbf{Implementation} \\
This chapter presents our implementation.
\item Chapter 4: \textbf{Results} \\
This chapter presents how our implementation performed.
\item Chapter 5: \textbf{Evaluation} \\
Here we summarize and evaluate the work presented in this report.

\end{itemize}
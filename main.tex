% Set a sane document class, 10pt font, and a report template
\documentclass[a4paper, twoside, openright, 12pt]{report}


% Import used packages
\usepackage{graphicx}
\usepackage{hyperref}
\usepackage{listings}
\usepackage{longtable}
\usepackage{lscape}
\usepackage{parskip}
\usepackage{color}
\usepackage{multirow}
%\usepackage[lmargin=25mm,rmargin=25mm,tmargin=40mm,bmargin=30mm]{geometry}
\usepackage{setspace}
\usepackage{fancyhdr}
\usepackage{multicol}
\usepackage[x11names, rgb]{xcolor}
\usepackage[utf8]{inputenc}
\usepackage{tikz}
\usetikzlibrary{decorations,arrows,shapes}
\usepackage{amsmath}



% Bibliographies
\usepackage[defernumbers=true]{biblatex}
\bibliography{bibliography}

% Use UTF-8
%\usepackage[utf8x]{inputenc}
\def \authors {Ken B\o{}rge Melhus Viktil \& Martin Tobias Holmedahl Sandsmark}
\def \papertitle {Jantu - A Cognitive Agent Playing StarCraft: Brood War} 

% Meta-information for the PDF
\hypersetup{
pdfauthor = \authors,
pdftitle = \papertitle,
pdfsubject = {Master project for IDI},
pdfkeywords = {project, cognitive, architecture, starcraft, artificial
    intelligence},
pdfcreator = {LaTeX with hyperref package lol},
pdfproducer = {pdflatex}}

%opening
\title{\papertitle}
\author{\authors}

\begin{document}

\begin{titlepage}
\begin{center}

\vspace*{5cm}
\hrule height 1pt
\includegraphics[width=30mm]{graphics/jantu.pdf}\\
\huge{\papertitle}

\vspace{.5cm}
\large{\authors}
\vspace{.5cm}
\hrule height 1pt

\vspace{6cm}
\end{center}
\normalsize
\begin{table}[!h]
\begin{tabular}{ll}
\multirow{4}{*}{\includegraphics[width=20mm]{graphics/logo.png}} & \\
& Department of Computer and Information Science \\
& Faculty of Information Technology, Mathematics and Electrical Engineering \\
& Norwegian University of Science and Technology \\
\end{tabular}
\end{table}
\vspace{.5cm}
\begin{center}
\today
\end{center}
\end{titlepage}

\thispagestyle{empty}
\cleardoublepage

\begin{abstract}
It has been shown that most players enjoys playing versus other human players over computer controlled opponents, but most artificial intelligence agents created today focus on winning in the most effective way possible. 
Cognitive architectures are designed to emulate how the human brain operates when performing tasks. Very little research has been done on applying cognitive methods to the field of real-time strategy games. 
In this paper we aim to research the use of a cognitive model that is capable of playing StarCraft: Brood War with decent results. The result is an AI agent implemented with a cognitive framework called LIDA. The resulting agent is only a proof of concept implementation, but we provide suggestions for how it can be improved in the future, and what the problems and limitations of the methods is. 
\end{abstract}
\thispagestyle{empty}
\cleardoublepage

\begin{abstract}
Bla bla samme som det engelske.
\end{abstract}
\thispagestyle{empty}
\cleardoublepage

\chapter*{Acknowledgements}
%%%%TOOOODOOO: SANDSMARK fix this shit to look better i guess
\thispagestyle{empty}

Firstly we would also like to thank our supervisors; Helge Langseth and Anders Kofod-Petersen. We would also like to thank the Cognitive Computing Research Group at the University of Memphis, for their LIDA framework which is the basis of our project. Finally we would like to thank the developers of the BWAPI project, and in particular Adam Heinermann, as well as the rest of the members in the \#BWAPI IRC channel on Freenode.

\clearpage

\cleardoublepage

\pagenumbering{roman}

\tableofcontents
\cleardoublepage
\listoffigures
\cleardoublepage
%\listoftables
\thispagestyle{empty}
\cleardoublepage

\pagestyle{fancy}
\pagenumbering{arabic}

%!TEX root = main.tex

\chapter{Introduction}
In this chapter we introduce our project. Section \ref{sec:background} presents the background and our motivation for the project and describes what we will do in this project and our main goals that we will accomplish. Section \ref{sec:name} is a short description of the where the name for the agent comes from. Section \ref{sec:structure} lays out the structure of this report.

\section{Background and Motivation}
\label{sec:background}

\subsection{The Problem}
Real-time strategy(RTS) games have been a complex problem to solve for artificial intelligence(AI) research for several years now. Ever since Michal Burro published an article in 2003\cite{buro2003real}, where he encouraged researchers to focus more of their expertise on the RTS domain, a lot of new methods have been applied to these problems with varying degrees of success. Some have been very successful and are now able to beat top level human players in real-time matches.\cite{campbell2002deep}

After Burro's paper several research platforms emerged, including OpenRTS\cite{buro2003orts} and Wargus\cite{wargus}. Wargus is an open source WarCraft 2 clone, which uses the Lua scripting language for its agent creation, while OpenRTS is a platform originally created for AI research. These platforms have some limitations, however, because they are mostly developed by hobbyists and not professional game development companies, they lack both polish and wide-spread usage outside of academic environments. They also lack a lot in the way of individual unit management, path finding and generally the possibilities for complex strategies and tactics, because of the very simple game mechanics.

As an answer to this the Brood War Application programming interface(BWAPI) was created. It supplies a comprehensive API for interacting with popular RTS-game StarCraft Brood War. The API lets developers easily create agents that play the game and can be run in tournaments against each other as well as against real players. It has gained a lot of popularity during the last years.\cite{bwapi} There are several competitions held each year, where programmers pit their BWAPI-based agents against each other, both for money and fame.\cite{sscait} But these artificial agents still have some way to go before they can match human players.\cite{eisbotvsfong}

While just winning the game is an interesting and worthy goal, it is according to Arrabales et al much more engaging for human players to play with other humans than with artificial agents.\cite{arrabales2009gamechars} We therefore think that developing human-like intelligence for playing computer games is something that should have a greater focus then it has had so far. Cognitive architectures are based around trying to emulate how the human consciousness works and acts in certain situations. There has been done very little research so far into using cognitive agents for playing RTS games, but they have been utilized with great success in first person shooter games previously.\cite{arrabales2009gamechars}

\section{The Project}
In this project we will implement a proof of concept cognitive agent that plays StarCraft: Brood War. We will be using a cognitive framework called LIDA in our implementation. The first step in the project will be to integrate the LIDA framework with StarCraft and BWAPI, so the LIDA cognitive framework can control the StarCraft process. We also need to create a bridge between the in-game time reference called frames, and the frameworks time references called ticks. After accomplishing this we should be able to control the execution of StarCraft from our framework, so we can pause, resume and step frame by frame and tick by tick for debugging purposes.

Next step will be to create the proof of concept implementation of a cognitive bot that can play the game, with decent results. This will allow us to investigate what kind of challenges present themselves when one implements an agent like this, and where there is room for improvement, and what the future might look like.

\section{The Name}
\label{sec:name}
The name {\em Jantu} is hindi, and is translated as ``Pertaining to the merely sentient part of a creature, as distinguished from the intellectual, rational, or spiritual part; as, the animal passions or appetites.''\cite{hindijantu}, which reflects what we are trying to achieve with our work in this project.

\section{Report Structure}
\label{sec:structure}
This report is structured into five main chapters:
\begin{itemize}
\item Chapter 1: \textbf{Introduction} \\
This chapter describes the background and motivation, as well as what we hope to accomplish during the project. It also layouts the general structure of the report.
\item Chapter 2: \textbf{Theory} \\
In this chapter we presents the relevant theory for understanding the tools and technologies used during the project. We presents some general information about StarCraft as a game, we also present some theory on the cognitive architectures in general, and more detailed about the framework we will be using.
\item Chapter 3: \textbf{Implementation} \\
In this chapter we present how we integrated the framework with StarCraft and what features we implemented for our agent.
\item Chapter 4: \textbf{Results} \\
In this chapter we present the results of running the agent, how well it performed and some insight into how the bot ended up interpreting and acting in the given environment.
\item Chapter 5: \textbf{Evaluation} \\
Here we summarize and evaluate results and the project. We discuss what the agent did good and why it had problems in the situations it did. We also outline how we think the agent can be improved in the future.

\end{itemize}

%!TEX root = main.tex

\chapter{Theory}
In this chapter we will the present background theory that is important for our project. In Section \ref{sec:starcrafttheory}, we have a general explanation of the game, and also explain the BWAPI architecture and how it utilizes shared memory and JNI to allow bots written in Java. Section \ref{sec:cogarch} describes models of cognition, the theory behind them and some architectures/implementations. Section \ref{sec:lida} describes the LIDA model of cognition as well as the LIDA framework that implements this model computationally.

%!TEX root = main.tex

\section{StarCraft}
\label{sec:starcrafttheory}


\begin{figure}[h!tb]
\centering
\includegraphics[scale=0.5]{graphics/scbw.jpg}
\caption{StarCraft: Brood War}
\label{fig:scbwIntro}
\end{figure}
%General introduction to the game, with emphasis on the parts that we utilize most in our project

StarCraft is a multi-player, real-time strategy (henceforth referred to as RTS) game, with a heavy focus on both economic strategies, as well as effective management of individual in-game units. Brood War is the expansion pack for the original StarCraft game that introduced new units, maps and upgrades. The game has been praised for its emergent complexity even though the mechanics are simple to understand. Brood Wars has been played on a high competetive level for over 10 years, and it has been evolving all that time with new strategies and tactics.

\subsection{Races}
The game has three races that a player selects between. The insect race Zerg, the advanced alien race Protoss and the humans Terran. Each race has it's own units and buildings as well as different game mechanics. 

\subsubsection{Terran}
\begin{figure}[h!tb]
\centering
\includegraphics[scale=0.7]{graphics/terranicon.png}
\caption{The in-game logo of the Terran race.\cite{terranlogo}}
\end{figure}

The terran are human colonists originally from earth. This is the most balanced race, which relies on a mixture of large numbers and powerful units. They have great mobility in their biological armies, and  great defense and slow turtling in their mechanical units like tanks. 

The Terran worker is a Space Construction Vehicle (SCV), and unique for Terran is that is can repair any mechanical unit or buildings if they get damaged.

Another unique mechanic with Terran is that their buildings can lift of and fly around to reposition themselves.

Several of the buildings can also be upgraded with addons that are smaller buildings that are constructed and connected to the main building. Addons unlock new upgrades and units to be constructed from the building it is connected to. Using lift off buildings can swap addons after they are constructed, so that opens up more possible build order diversity.

Every Terran unit can be healed after they take damage, the mechanical units and buildings can be repaired using an SCV, but doing so means that the worker will not be gathering resources for that time. To heal damaged biological units you have to train special units called Medics. 

\subsubsection{Zerg}
\begin{figure}[h!tb]
\centering
\includegraphics[scale=0.3]{graphics/zergicon.png}
\caption{The in-game logo of the Zerg race.\cite{zerglogo}}
\end{figure}

The zerg is an insect-like collection of different biological races assimilated under a central intelligence.

They rely on biological abilities that are selectively evolved as appose to using technology like the other two races.

The individual units in the Zerg army is not very powerful, but their strength lies in superior numbers of units as well as the ability to quickly reinforce the army with new units when some dies off.

The Zerg worker unit is called a drone, and what separates this worker from the other races is that in order to create a new building the worker has to sacrifice it self to morph into the new building.

Zerg has a unique game mechanic for creating new units. The main building, called the hatchery, creates larva on regular intervals, up to a maximum of three, that can be morphed into new units. Both workers and army units are created from the larva, and because of the limited availability on these a player has to prioritize at all time if he wants to create workers or army units.

To increase the production capacity they have to either expand to a new base, or create several hatcheries in their current bases to gain access to more larva.

Zerg can only build new buildings on creep, something that spreads from existing buildings as well as Creep Colonies, a special building that is used only for extending creep coverage on the map.

Zerg units have a unique ability to regenerate health after they are damaged if they get left alone out of combat. This makes retreating and regrouping a good tactic for zerg players as their army will have time to regenerate back to full health after a defeat on the battleground. Together with another unique ability some of the Zerg units have that is called burrow, that allows the to burrow under ground and hide or run away, it can be really hard to kill some units since they can escape and regenerate the lost health. 

\subsubsection{Protoss}
\begin{figure}[h!tb]
\centering
\includegraphics[scale=0.3]{graphics/protossicon.png}
\caption{The in-game logo of the Protoss race.\cite{protosslogo}}
\end{figure}

The protoss is a highly advanced race with powerful mental abilities.

They are both technologically and military advanced, and usually rely on few, but very powerful single units. They have very expensive units that can crush a much bigger army by themselves.

The protoss worker is called a probe, probes doesn't need to construct the buildings, they only tag an area and then the building gets warped in from the protoss home world. This means a single probe can start the construction of several buildings at the same time and then return to gathering resources.

Similar to Zerg, Protoss can't simply construct buildings anywhere, they have to be constructed on a power grid that is generated by pylons, the Protoss supply building.

A unique Protoss feature is that all the units and buildings have shields that protect them from damage and regenerates over time. For an enemy to damage the unit it first has to deplete the shield, then it can do damage to the health of the unit. But damage taken after the shield is down cannot be healed or repaired.

\subsection{Gameplay}
\subsubsection{Micro vs. macro}
Two well-known concepts in the RTS communities, and the StarCraft community in particular, are micro management and macro management. Macro management refers to large-scale economic and strategic decisions, while micro management refers to smaller-scale control of individual units, or groups of units.

A good control of both concepts is needed for a successful agent.

\subsubsection{Supply}
Supply is a term for an artificial limit imposed by the game on how many units a player is allowed to make at any time. To increase the supply, a player can build a special kind of units; for the Protoss race this is {\em pylons}, for Zerg it is {\em overlords} and for Terran it is {\em supply depots}. The name stems from the terran unit needed, supply depots.

\subsubsection{Fog of war and scouting}
Fog of war is a well-known term from RTS games, which denotes un-observed parts of the environment or map. In StarCraft this is shown as shaded on screen. This leads to partial observability, and gives rise to uncertainty about the rest of the map, about what the other player is doing and what resources he has exhausted. To counteract this, it is common to {\em scout}, that is, to send out units to simply observe the other player.

\subsection{BWAPI}
To ease the development of third-party agents that play the game, an application programming interface (API) has been developed for StarCraft: Brood War, by third-party developers. They have relied on reverse engineering of the original game for developing it. It works by injecting itself into the process of the game, and hooking into various functions used in the game, as well as reading various memory areas directly.\cite{bwapi}

There has sprung up a sizable community around this effort, and there are several tournaments where programmers can participate with their own agents.\cite{bwapi}

There are also several third-party addons and extra libraries for easing the development of agents, such as the Brood War Terrain Analyzer (BWTA), which provides easy-to-use functions for analyzing the maps for finding choke-points and suitable locations for various buildings,\cite{bwta} and the Brood War Standard Addon Library (BWSAL) which is both a generic, modular framework for BWAPI agents as well as default implementations for a large part of the modules needed for implementing such an agent.\cite{bwsal}

BWAPI only provides a C++ API, so for using it from other languages various types of bindings are needed.

There are two modes for loading agents using BWAPI; loading it directly into the StarCraft process, or using a shared-memory area to communicate state between the agent process and the StarCraft process in which the BWAPI code is running.\cite{bwapi}

\subsubsection{JNIBWAPI}
For using Java for developing an agent the most well-supported is using the JNIBWAPI project, which uses JNI to provide Java-bindings for BWAPI. It utilizes the shared-memory approach of BWAPI to avoid having to load the Java virtual machine into the StarCraft memory.

\section{Cognitive Architectures}
\label{sec:cogarch}
Cognitive architectures are architectures that base themselves on some model of human cognition. There are several competing theories about how cognition works, and one of the most well-supported is the Global Workspace Theory.

There hasn't been widespread research done with cognitive architectures, however. Arrabales et al put forth that this is perhaps because of poor understanding of cognitive research amongst researchers in the field of classical AI. \cite{arrabales2009gamechars}

There is many different cognitive architectures available today\cite{duch2008cognitive}. We chose to focus on those based on the Global Workspace theory, for several reasons. It's widely accepted as the currently best model of understanding consciousness\cite{Franklin2012}, implementations using it generally provide believable results\cite{arrabales2009gamechars}\cite{snaider2011lida} and it has a solid grounding in neurological and psychological models, experiments and empirical evidence\cite{Franklin2012}.

\subsection{Global Workspace Theory}
\begin{figure}[h!tb]
\centering
\includegraphics[scale=1.0]{graphics/globalworkspace.png}
\caption{A schematic of the Global Workspace theory\cite{baars2005gwt}}
\label{fig:gwt}
\end{figure}

Global Workspace Theory is a model of cognition that is very well supported by experimental data, and is one of the most widely accepted models. \cite{dehaene2001towards} It has been used to implement processes that imitate human decision making (for example for solving the problem of assigning people to jobs in the US Navy).\cite{baars2005gwt}\cite{franklin2003interacting}

It is based around an understanding of the brain as a set of many independently processing modules, working together by utilizing a shared workspace (hence the ``global workspace''). Every ``cognitive cycle'' all the processes compete for attention, and a single one gets the proverbial spotlight shone upon it.\cite{baars2005gwt} When this happens, all the other modules/processes receive the ``broadcasted'' data from that module, and use it as they see fit; for example memory centers store the information received. This solves the relevancy problem\footnote{The relevancy problem is part of the frame problem, the problem of knowing what to apply where in problem solving, and was considered to might be intractable.\cite{shanahan2005applying}}, and it lets parts of the brain collaborate on problems that can't be solved unconsciously by individual modules.

See figure \ref{fig:gwt} for a graphical overview of the theory.

\subsubsection{The theatre metaphor}
A common metaphor used is the {\em theatre metaphor}. Here we view the mind as a proverbial theatre, with an audience and a stage with a spotlight shining on it.

Then various actors on stage get the spotlight shone upon them (according to the pandemonium theory which is often used; the one who shouts the loudest\cite{selfridge1958pandemonium}), and is allowed to share its message with the the audience.

In the dark, behind the stage, are the various executive processes (represented by script writers, directors, etc.) who aren't visible but still help form what is seen.

\subsubsection{Neurological basis for the Global Workspace Theory}
According to R. Llinás et al.\cite{llinas1998neuronal}, the looped neural pathways between the thalamus and the cortex might be what is responsible for the conscious collaboration between various parts of the brain. The thalamus is responsible for letting various parts of the cortex broadcast and influence the rest of the cortex. This meshes well with the global workspace theory, and the idea of a single subsystem broadcasting its contents to the rest of the whole.

There has also been discovered a link between the switching between coherent and decohoerent EEG-activity that seems to indicate a switching between states of competition for access to the global workspace. Decoherent electric activity seems to indicate a competitive process, while coherent activity indicates a passive ``listening''.\cite{freeman2003neurobiological}

The periods between these states is compatible with the widely accepted figures for the time it takes for stimulus to become conscious.\cite{shanahan2005applying}

Several studies support the theory that consciousness is what enables global activation. They usually compare conscious and unconscious conditions in conscious subjects, either by sensory stimulation or overpractice of automatic habits\footnote{Consciously doing tasks that usually are done unconsciously.}. All results show that conscious processes lead to widespread cortical activation, while unconscious ones usually only activates local regions.\cite{baars2003brain}

There has also been experiments done where subjects have been in various states of unconsciousness (sleep, general anasthesia, epileptic loss of consciousness and vegetative states), and it has been found that sensory stimulation in all states only lead to local activation in the cortex, which seems to indicate that conscousness leads to spreading activation, and collaboration between parts of the cortex.\cite{shanahan2005applying}

\subsubsection{Implementations}
There has been implementations of several architectures incorporating the global workspace theory. Here follows some of the most significant ones.

\paragraph{IDA}
In the mid-nineties an artificial agent was designed for the U. S. Navy that would replace so-called ``detailers'', which are specialists that allocate personell. They take into account personal preferences, moving costs, requirements from the Navy (number of personell on each boat etc.) and various regulations from the Navy. Preferences from sailors are taken in from natural language e-mails, and these are processed using a slipnet that processes the natural language into something that has usable semantic meaning for the system. The other inputs (e. g. regulations and requirements from the Navy) are stored in databases, and are pre-processed by similar slipnets before being pushed into the system. This system is able to replace the 300 or so detailers that the Navy employs. The system is up and running, and is matching the performance of the human detailers. \cite{baars2007architectural}\cite{franklin1998ida}

It was based on the Virtual Mattie agent, which was a virtual clerical agent, responsible for interacting with humans through natural language over e-mail.\cite{franklin1996virtual}

\paragraph{LIDA}
One problem with IDA was that learning wasn't really implemented at all, and therefore they started re-implementing from scratch, in a project called LIDA; the Learning IDA system. The idea was to add learning from experience; learning newly perceived objects and their relationships to already known objects, relationships between objects, categories, relationships between objects and actions, effects of actions, and improved perception/tagging of sensory data with learned memories.\cite{franklin2006lida} Over time, however, LIDA evolved to become a more generic Java-framework for cognitive architectures.\cite{snaider2011lida}

\paragraph{CERA-CRANIUM}
\begin{figure}[h!tb]
\centering
\includegraphics[width=\textwidth]{graphics/ceracranium.png}
\caption{Overview of the CERA-CRANIUM architecture.}
\label{fig:cera-cranium}
\end{figure}
This is a two-fold architecture, as reflected in the name. {\em CRANIUM} is a more generic tool for creation of a high amount of parallel processes operating with shared workspaces. {\em CERA} uses these services for creating a dynamic and adaptable system which operates on perceptions and generates actions, based on a cognitive model.\cite{arrabales2009gamechars} It has been used for making agents that act in several different environments, both virtual environments like the computer game Unreal Tournament, as well as real environments, where it has been embodied in small robots that map out unknown environments. \cite{arrabales2009ceracranium} A high-level overview of the architecture can be seen in Figure \ref{fig:cera-cranium}.

\subsection{Cognitive Models in Game AIs}
There have been several implementations of models of cognition into game-playing agents. Examples of this is CERA-CRANIUM, which was used to implement an agent playing Unreal Tournament \cite{arrabales2009ceracranium}, and SORTS which implemented an agent for the real-time strategy game ORTS using the symbolic, cognitive architecture Soar.\cite{wintermute2007sorts} One of the reasons for using computer games for experiments with regards to high-level artificial intelligence is that the characteristics of computer games lend themselves to this, by eliminating noise and uncertainty, and providing a more or less realistic simulated environment.

\section{LIDA Framework}
\label{sec:lida}
LIDA is both the name of a cognitive model and a software framework implementing
most parts of the LIDA model.
Explain the important parts of the framework and how they are used and implemented

%!TEX root = main.tex

\chapter{Implementation}
In this chapter we describe how we implemented the project.
Section \ref{sec:integration} how we integrated bwapi with lida
Section \ref{sec:detectors} feature detectors we made
Section \ref{sec:actionexecution} how action are executed from lida to bwapi



\section{Integration}
\label{sec:integration}
BWAPI itself provides only a basic C++ API, so JNIBWAPI provides, as explained in chapter \ref{sec:starcrafttheory}, a custom Java API using the Java Native Interface, JNI.\cite{jni} But because we can't load an entire Java VM into the StarCraft process, we use shared memory to connect the StarCraft process with the process of our agent. We found that to do this you need to run the processes with Administrator privileges in Windows.

We also updated JNIBWAPI to adapt to some minor recent changes in BWAPI, and added in some missing functionality that we needed for controlling the game, like starting, pausing and restarting the on-going game.

But to use the LIDA framework this has to be integrated with JNIBWAPI. So to accomplish this we implement the domain specific modules of the LIDA framework to make calls to JNIBWAPI. 


\begin{figure}[h!tb]
\centering
\includegraphics[scale=1.0]{graphics/jantu.png}
\caption{A general architecture overview of Jantu}
\label{fig:jantu}
\end{figure}

Figure \ref{fig:jantu} shows a general overview of the Jantu bots architecture, which consists of 3 main parts. In the Starcraft process the game it self runs together with BWAPI injected into the game client. Since this is all running in c++ and we are using the Java interface for BWAPI our code is run in a separate process that communicates with the Starcraft process using a shared memory bridge, this enables JNI-BWAPI to make calls to BWAPI and retrieve information back across the bridge. 

In the Jantu process JNI-BWAPI is running together with the LIDA framework. In order to integrate the LIDA framework with JNI-BWAPI we setup the framework first with the configurations we needed to get it up and running. This involves describing and structuring the modules you need in different XML configuration files. Also in these files we configure what kind of information that will be possible to use and transmit internally in the LIDA framework. 

JNI-BWAPI consists of different models and types that represent Starcraft information like units and buildings together with a lot of native functions that can be called to communicate with BWAPI. So we integraded this with LIDA by using a custom implementation of the Environment class in LIDA. This class becomes the interface between the domain specific modules of LIDA, the sensory memory and sensory-action memory, and JNI-BWAPI. 


\subsubsection{Environment module}
The Environment module is what is between the JNI-BWAPI and LIDA. It is responsible for abstracting away the JNIBWAPI, and making sure that the game runs when it should. It allows LIDA to reset the state of the environment, by restarting the game.
\paragraph{Timing} To be able to use the run/pause and timing functionality of LIDA inside StarCraft, we have a semaphore that is released by a LIDA codelet each tick. Then the callback from StarCraft/BWAPI waits for this to be released. This allows us to easily set the amount of StarCraft frames that are processed for each LIDA cycle, by increasing the amount of permits available in the semaphore. In our implementation we let one cognitive cycle equate to one in-game frame, however.
\begin{figure}[h!tb]
\centering
\includegraphics{graphics/environment-gui.png}
\caption{Our custom representation of the state of the environmen}
\label{fig:environment-gui}
\end{figure}
\paragraph{GUI panel} We also provide a custom GUI panel to represent the environment, see Figure \ref{fig:environment-gui}. Different regions from the Brood War Terrain Analyzer are separated by gray lines. Enemy entities are displayed as red dots, entities owned by our agent are blue dots. Neutral entities (vespene geysers and mineral fields) are green. Choke points are marked by yellow circles.


\section{Detectors}
\label{sec:detectors}
Feature detectors is how LIDA perceives it's environment and identifies important aspects of the current game state. They are task that are run at specific intervals and parses the game state at that time in order to identify a given feature that can later be used in different modules in LIDA recognize thoughts and concepts. Each detector usually only identifies one specific feature, but it is possible for one detector to identify several features of they are of the same type. 

These are the detectors we implemented: 
\begin{itemize}
\item \textbf{IdleWorkerFeatureDetector} \\
This feature detector identifies worker units that doesn't have a job, a worker could be gathering resources, constructing buildings or scouting. But for efficiency it should be doing something at all times. 
\item \textbf{LarvaFeatureDetector} \\
This feature detector identifies larvae that are ready to be morphed into units. This is the only way Zerg can create units and when no larvae are available they can't create any more units until more larvae spawn. 
\item \textbf{ResourceFeatureDetector} \\
This feature detector identifies what type of units and buildings that we currently have enough resources available to create. This can be buildings we can construct, units we can morph or upgrades that we can research. 
\item \textbf{SupplyBlockFeatureDetector} \\
This feature detector identifies when we are getting close to being supply blocked, that means that we can't build any more units because another supply-granting building/unit is created.
\item \textbf{UnsaturatedResourcesFeatureDetector} \\
This feature detector identifies whether or not our available resource nodes have are saturated with enough workers that are gathering them. This can be used to decide if we need to build more workers or not.
\end{itemize}

Feature detectors can be created that detect almost every aspect of the game, and they can be everything from simply detecting the existence of specific units or game elements to more complex detectors that detect army compositions or enemy tactics and strategies. The more of them you implement the more advanced concepts you can identify and that opens up more advanced strategies you can perform yourself. 



\section{Action Execution}
\label{sec:actionexecution}
Action are executed with commands!

%!TEX root = main.tex

\chapter{Results}
This chapter presents results of how our implementation performed.
\section{Architecture}
\label{sec:architecture}



\subsection{Bot testing}
Given any testing of the integrated bot, mabye in several stages, vs either other bots or the built in AI in starcraft. How did it performe

%!TEX root = main.tex

\chapter{Evaluation}
In this chapter we conclude our work by looking at the goals defined in the
introduction, and evaluate the results.
Section \ref{sec:evalres} evaluates the results of the project.
Section \ref{sec:conclusion} contains the final conclusion for the project and this report. 
Section \ref{sec:futurework} contains our thoughts on future implementation and improvements of our solution.



\section{Evaluation of Results}
\label{sec:evalres}
Our motivation for this project was to create an agent, using LIDA, that could play StarCraft: Brood War, which we succeeded in. We also discovered some of the problems inherent in the use of this cognitive architecture for this domain.

\section{Conclusion}
\label{sec:conclusion}
We succeded in achieving what we set out to do, and while the performance was not excellent, we think it is a very good starting point for any future work in this area.

\section{Future Work}
\label{sec:futurework}
We believe that there are several areas to be worked on. Implementing more feature detectors, and more behaviours, might be the easiest and obvious way to go forward. Replacing the rule-based action selection module with a more advanced approach is also something to look into. Version 1.2 of the LIDA framework replaces the default rule-based action selection with a behaviour network, based on Maes' behaviour network.\cite{maes1989right}

A more significant change would be to use the various memory modules available more actively.\cite{franklin2007lida}. This could help with for example letting the bot remember more about the state of the game, for example if it has already built the necessary building necessary for a certain unit, or remember build orders (learned from earlier games).

Learning could also be implemented at several other levels, not just as long-term memory.
\endrefsection

\cleardoublepage
\phantomsection
\addcontentsline{toc}{chapter}{Bibliography}
\printbibliography[segment=0]

\end{document}
